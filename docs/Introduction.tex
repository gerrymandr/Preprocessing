This document is written to be a brief explanation of two 
processes which we call {\it proration} and 
{\it roundoff}. 
{\it Proration} is the process of taking electoral data given in 
terms of one partition and assigning it to another partition of a 
state. For example, if election data is given in 
terms of old precincts, and we want to see it in terms of 
current precincts, which might be geographically different 
from the past precincts.
\newline
~\begin{center}
  \resizebox{0.4\textwidth}{!}{
  \begin{tikzpicture}[scale=0.5]
    \draw[\thicc, xstep = 6, ystep = 4] (0,0) grid (12,12);
    \node (A) at (6,-1) [scale = 1.3, draw, \thicc, rectangle] {Election data in old precincts};
    \draw[\thicc, color = Plum, xstep = 4, ystep = 6] (16,0) grid (28,12);
    \node (B) at (22,-1) [scale=1.3, color = Plum, draw, \thicc, rectangle] {New precincts};
    \node (C) at (14,-2) {};
    \draw[xshift = 2cm, yshift = 1cm, \thicc, xstep = 6, ystep = 4] (6,-16) grid (18,-4);
    \draw[yshift = 3cm, \thicc, color = Plum, xstep = 4, ystep = 6] (8,-18) grid (20,-6);
    \node (D) at (14,-16) [scale = 1.3, draw, \thicc, rectangle] {Convert election data to new units};
    \draw [->, \thicc] (A) to [out = -90, in = 180] (D) {};
    \draw [->, \thicc] (B) to [out = -90, in = 0] (D) {};
  \end{tikzpicture}
  }
\end{center}
{\it Roundoff} is the process of assigning small units to larger 
units that may not match. For example, if district lines cut 
through counties, and you want to assign counties to 
districts, roundoff is a way to make the assignment in a 
(mostly) deterministic and complete manner.
